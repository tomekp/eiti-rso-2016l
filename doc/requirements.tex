\documentclass[a4paper,11pt]{article}

\usepackage{polski}
\usepackage[utf8]{inputenc}
\usepackage{graphicx}
\usepackage{enumitem}
\usepackage{listings}

%opening
\author{
  Tomasz Pęksa \\
  Paweł Stiasny
}
\date{28 kwietnia 2016}


\begin{document}

\makeatletter
\begin{titlepage}

\begin{center}
  \includegraphics[scale=0.1]{Logo_PW_black.png} \\
  \vspace{50pt}
  
  {\LARGE Usługa bezpiecznej, niezawodnej dystrybucji przetworzonej, chronionej informacji} \\
  \vspace{10pt}
  
  {\large Projekt realizowany w ramach przedmiotu RSO} \\
  \vspace{40pt}
  
  {\Huge Wymagania, architektura, harmongoram} \\
  \vspace{30pt}
  
  {\@author} \\
  \vspace{10pt}
  
  {\@date}
\end{center}

\end{titlepage}
\makeatother

\tableofcontents
\newpage

%%%%%%%%%%%%%%%%%%%%%%%%%%%%%%%%%%%%%%%%%%%%%%%%%%%%%%%%%%%%%%%%%%%%%%%
\part{Specyfikacja wymagań}

Priorytety wymagań zostały zdefiniowane w skali (Niski, Średni, Wysoki).
Posłużą one do ustalenia kolejności implementacji zadań.

\section{Wymagania funkcjonalne}
Wymagania funkcjonalne są opisane w postaci historyjek użytkownika ze
scenariuszami testowymi w myśl metody BDD.  

\subsection{Perspektywa użytkownika}
\label{userPerspective}

\subsubsection{Pobranie agregatu danych}
Jako użytkownik chcę pobierać część wspólną linijek z podanych przez
siebie plików przechowywanych w usłudze aby uzyskać dostęp do
przetworzonej chronionej informacji.

\textbf{Priorytet}: Wysoki

Niech w usłudze będą przechowywane pliki o znanych użytkownikowi
identyfikatorach.

Kiedy użytkownik zażąda pobrania agregatu plików o wybranych
identyfikatorach

Wtedy usługa dostarcza na lokalną maszynę użytkownika plik składający
się z części wspólnej linijek wybranych plików.


\subsection{Perspektywa administratora}

\subsubsection{Inicjalizacja danych w usłudze}
Jako administrator chcę pobrać zasoby z zewnętrznego źródła do plików
przechowywanych w usłudze aby umożliwić użytkownikom korzystanie z nich.

\textbf{Priorytet}: Wysoki


Niech pod pewnym adresem URL istnieje zasób dostępny przez HTTP lub
HTTPs.

Kiedy użytkownik wywoła z linii poleceń komendę klienta ,,\texttt{rsoctl
fetch <url>}'',

Wtedy klient zwraca identyfikator pliku, który użytkownik może użyć do
pobrania pliku z usługi.

\subsubsection{Dodanie węzła warstwy wewnętrznej}
Jako administrator chcę dodać nowy węzeł magazynowy aby zwiększyć
pojemność systemu.

\textbf{Priorytet}: Niski

Zakładając, że konfiguracja zawiera już jakąś listę węzłów warstwy
wewnętrznej,

Kiedy administrator doda do niej nową pozycję

Wtedy usługa pozwala nowemu węzłowi na dołączenie do sieci, zachowując
spójność przechowywanych danych.

\subsubsection{Dodanie węzła warstwy zewnętrznej}
Jako administrator chcę dodać nowy węzeł dostępowy aby zwiększyć przepustowość systemu.

\textbf{Priorytet}: Niski

Zakładając, że konfiguracja zawiera już jakąś listę węzłów warstwy zewnętrznej,

Kiedy administrator doda do niej nową pozycję

Wtedy nowy węzeł daje użytkownikom dostęp do usługi równoważny z dostępem przez pozostałe węzły.

\subsubsection{Uruchomienie węzłów usługi}
Jako administrator chcę dodać nowy węzeł dostępowy aby zwiększyć przepustowość systemu.

\textbf{Priorytet}: Niski

Zakładając, że konfiguracja zawiera już jakąś listę węzłów warstwy zewnętrznej,

Kiedy administrator doda do niej nową pozycję

Wtedy nowy węzeł daje użytkownikom dostęp do usługi równoważny z dostępem przez pozostałe węzły.

\subsubsection{Zatrzymanie węzłów usługi}
Jako administrator chcę móc zatrzymać usługę na wszystkich węzłach jedną komendą aby nie musieć zarządzać każdym węzłem indywidualnie.

\textbf{Priorytet}: Niski

Zakładając, że usługa jest uruchomiona

Oraz opisane w konfiguracji węzły są dostępne w sieci

Kiedy administrator wywoła na węźle komendę ,,\texttt{rsoctl stop}''

Wtedy usługa zostanie zatrzymana na wszystkich węzłach zdefiniowanych w konfiguracji

\subsubsection{Sprawdzanie stanu węzłów usługi}
Jako administrator chcę znać stan wszystkich węzłów usługi aby być w
stanie rozpoznać problemy i wiedzieć, czy konieczne jest podjęcie
działania w celu ich rozwiązania.

\textbf{Priorytet}: Wysoki

Zakładając, że usługa jest uruchomiona

Oraz opisane w konfiguracji węzły są dostępne w sieci

Kiedy administrator wywoła na węźle komendę ,,\texttt{rsoctl status}''

Wtedy zostaną zebrane i wyświetlone informacje o stanie wszystkich
węzłów usługi.

\section{Awarie}
W czasie działania usługi może wystąpić szereg awarii. Usługa powinna
pozostać dostępna, tzn. kontynuować poprawne świadczenie funkcjonalności
opisanych w poprzedniej sekcji w przypadku wystąpienia tych awarii oraz
ich kombinacji. Odporność na awarie może być testowana przez
wstrzykiwanie usterek w trakcie przebiegu podstawowych testów
funkcjonalnych.

Niech:

\[ n = \mbox{liczba węzłów magazynowych} \]
\[ i, j, k \leq \mbox{liczba replik podana w konfiguracji} - 1 < n \]

\subsection{Awaria węzłów}
Następuje nieplanowane wyłączenie $k$ z $n$ węzłów warstwy wewnętrznej.

\textbf{Priorytet}: Średni

\subsection{Awaria połączeń pomiędzy warstwą zewnętrzną i wewnętrzną}
Węzeł warstwy zewnętrznej nie może połączyć się lub traci nawiązane
połączenie z $i$ kolejnymi węzłami warstwy wewnętrznej.

\textbf{Priorytet}: Średni

\subsection{Awaria połączeń pomiędzy węzłami warstwy wewnętrznej}
Węzeł warstwy wewnętrznej nie może połączyć się lub traci nawiązane
połączenie z $j$ kolejnymi węzłami warstwy wewnętrznej.

\textbf{Priorytet}: Średni


\section{Działania niepożądane}
\label{attackVectors}
Wymagane jest, żeby usługa była niepodatna na następujące wektory ataku:

\subsection{Włączenie nieuprawnionego węzła}
Atakujący może próbować podstawić pod czasowo nieosiągalny (na przykład
w wyniku celowego przerwania łączności lub utraty dostępności w wyniku
ataku typu Denial of Service) węzeł własny, wrogi węzeł i w ten sposób
uzyskać kontrolę nad systemem lub uzyskać dostęp do chronionej
informacji. Nie rozważamy sytuacji, gdy administrator usługi
współpracuje (świadomie i dobrowolnie bądź nie) z atakującym.

\textbf{Priorytet}: Średni

\subsection{Dostęp przez nieautoryzowanego użytkownika}
Atakujący może próbować uzyskać dostęp do którego nie ma uprawnień z
poziomu zewnętrznego klienta. Może próbować to osiągnąć: polegając
bezpośrednio na niezabezpieczonym API usługi; monitorując ruch pomiędzy
uprawnionym użytkownikiem i usługą; uzyskując nieautoryzowany dostęp do
maszyny uprawnionego użytkownika. Nie rozważamy sytuacji, gdy uprawniony
użytkownik usługi współpracuje (świadomie i dobrowolnie bądź nie) z
atakującym.

\textbf{Priorytet}: Niski


%%%%%%%%%%%%%%%%%%%%%%%%%%%%%%%%%%%%%%%%%%%%%%%%%%%%%%%%%%%%%%%%%%%%%%%
\part{Architektura systemu}
Poniżej przedstawiamy zarys planowanej przez nas architektury.
Szczegóły zostaną rozwinięte w kolejnej iteracji i mogą ulec zmianie.

\section{Model danych}
\subsection{Plik}
W usłudze przechowywane mogą być dowolne dane tekstowe.  Dane te są
trzymane w plikach.

\subsection{Identyfikator pliku}
Identyfikatory są przydzialane przez usługę przy tworzeniu plików.
Są to dowolne wartości liczbowe.

\subsection{Identyfikator węzła}
To każdego węzła warstwy wewnętrznej przypisany jest numeryczny
identyfikator w zakresie pomiędzy 1 a liczną węzłów.

\section{Kluczowe elementy infrastruktury}
\subsection{Węzły warstwy wewnętrznej}
Warstwa wewnętrzna odpowiada ze magazynowanie danych.  Proces pracujący
na węzłach tej warstwy udostępnia następujące API:
\paragraph{Dostęp do plików}
\begin{itemize}
  \item Pobierz plik składowany na węźle
  \item Ściągnij zewnętrzny zasób do pliku
\end{itemize}
\paragraph{Operacje składowe replikacji}
\begin{itemize}
  \item Pobierz plik z innego węzła i zapisz jako replikę
  \item Usuń replikę
  \item Lista plików składowanych na węźle
\end{itemize}

\subsection{Węzły warstwy zewnętrznej}
Warstwa zewnętrzna odpowiada za:
\begin{itemize}
  \item Pobieranie informacji z warstwy wewnętrznej (magazynowej)
  \item Przetwarzanie informacji
  \item Udostępnianie przetworzonej informacji klientom usługi
  \item Monitorowanie żywotności warstwy wewnętrznej
\end{itemize}

Proces działający w warstwie zewnętrznej w swoim API udostępnia
następujące operacje:
\begin{itemize}
  \item Pobierz plik z zewnętrznego zasobu
  \item Przekaż treść pliku
\end{itemize}

%%%%%%%%
%
%   Szczegóły API do sklejenia z opisu Tomka:
%      https://gist.github.com/tomekp/c0037a7d2e45259fb39d
%      https://gist.github.com/pstiasny/8c6b447c2c43565fbaf390aad355cfb2
%
%%%%%%%%

\subsection{Klient}
Na maszynie użytkownika końcowego instalowany będzie klient z
interfejsem linii poleceń.  Jego zadaniem będzie komunikowanie się z
węzłami warstwy zewnętrznej w celu realizacji celów użytkownika
opisanych w sekcji \ref{userPerspective}.

\section{Mechanizm replikacji}
Mechanizm opiera się w dużej mierze na indepotentności operacji
udostępnianych przez API warstwy wewnętrznej.  Dzięki tej właściwości
usługa kontrolera replikacji może zostać wydelegowana do węzłów warstwy
zewnętrznej i funkcjonować poprawnie nawet kiedy uruchomiona jest
równocześnie na wielu węzłach.  Dzięki temu, że każdy z węzłów warstwy
zewnętrznej bada dostępność odpowiedniej liczby replik, takie
potraktowanie problemu zapewnia dostępność nie tylko w sytuacji awarii
$r$ węzłów, gdzie $r$ -- liczba replik, ale także $r$ połączeń warstwa
zewnętrzna - warsta wewnętrzna.

\begin{lstlisting}
Plik { idZasobu, zadanaLiczbaReplik }
Wezel w warstwy wewnetrznej udostepnia
  listaPlikow(w) =
    lista plikow przechowywanych na wezle
  pobierzReplike(w, wZrodlowy, idZasobu) =
    pobierz zasob z wezla zrodlowego i zapisz jako replike
  usunReplike(w, idZasobu) =
    usun (nadmiarowa) replike z lokalnego magazynu

KontrolerReplikacji.zapewnijDostepnosc() =
  repliki = pusta mapa Plik -> zbior Wezlow
  for idWezla in [0, liczbaWezlow)
    w = Wezel(idWezla)
    lp = listaPlikow(w)
    for plik in lp
      dodaj w do repliki[plik]
  for plik -> wezly in repliki
    while liczba(wezly) < zadanaLiczbaReplik(plik)
      kandydaci = znajdzWezlyDlaId(idZasobu(plik))
      w = wez wezel z kandydaci
      if w not in wezly and dostepny(w)
        pobierzReplike(w, dowolny wezel z wezly, idZasobu(plik))
        dodaj w do wezly
    if liczba(wezly) > zadanaLiczbaReplik(plik)
      wez nadmiarowe wezly doUsuniecia ostatnie wzgledem porzadku
        okreslanego przez znajdzWezlyDlaId
      for w in doUsuniecia
        usunReplike(w, idZasobu(plik))
\end{lstlisting}

Należy zauważyć, że przy równoczesnym działaniu wielu kontrolerów
replikacji może dojść do migotania replik w sytuacji, gdzie jeden z nich
,,widzi'' pewną replikę, tj. ma sprawne połączenie do węzła ją
przechowującego i zlecał usunięcie nadmiarowych replik, a inny nie.
Aby uniknąć tego zjawiska implementacja operacja \texttt{usunReplike}
powinna czekać z faktycznym usunięciem przez okres znacząco dłuższy
od okresu badania żywotności i odwoływać je gdyby pojawiło się ponowne
żadanie pobrania repliki.


\section{Sieć}
Komunikacja pomiędzy węzłami oraz węzłami i klientami odbywać się będzie
za pomocą protokołu HTTPS.  Adresy poszczególnych węzłów warstwy
będą zapisane w pliku konfiguracyjnym usługi.  Jeśli będzie to
konieczne, możliwe będzie też rozszerzenie funkcjonalności o dynamiczne
wykrywanie węzłów.

Wszystkie połączenia pomiędzy węzłami oraz węzłami i klientami będą
nawiązywane bezpieczenie z pośrednictwem protokołu TLS.  W celu
realizacji wymagań opisanych w sekcji \ref{attackVectors} obie strony
połączenia będą uwierzytelniane z użyciem kluczy kryptografii
asymetrycznej z użyciem metod wbudowanych w protokół.  Klucze publiczne
węzłów i użytkowników zostaną zapisane w konfiguracji usługi.

\end{document}
