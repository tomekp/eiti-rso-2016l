\documentclass[a4paper,11pt]{article}

\usepackage{polski}
\usepackage[utf8]{inputenc}
\usepackage{graphicx}
\usepackage{enumitem}

%opening
\author{
  Paweł Stiasny
}
\date{28 kwietnia 2016}


\begin{document}

\makeatletter
\begin{titlepage}

\begin{center}
  \includegraphics[scale=0.1]{Logo_PW_black.png} \\
  \vspace{50pt}
  
  {\LARGE Usługa bezpiecznej, niezawodnej dystrybucji przetworzonej, chronionej informacji} \\
  \vspace{10pt}
  
  {\large Projekt realizowany w ramach przedmiotu RSO} \\
  \vspace{40pt}
  
  {\Huge Specyfikacja wymagań} \\
  \vspace{30pt}
  
  {\@author} \\
  \vspace{10pt}
  
  {\@date}
\end{center}

\end{titlepage}
\makeatother

\tableofcontents
\newpage

\part{Specyfikacja wymagań}

Priorytety wymagań zostały zdefiniowane w skali (Niski, Średni, Wysoki).
Posłużą one do ustalenia kolejności implementacji zadań.

\section{Wymagania funkcjonalne}
Wymagania funkcjonalne są opisane w postaci historyjek użytkownika ze
scenariuszami testowymi w myśl metody BDD.  

\subsection{Perspektywa użytkownika}

\subsubsection{Pobranie agregatu danych}
Jako użytkownik chcę pobierać część wspólną linijek z podanych przez
siebie plików przechowywanych w usłudze aby uzyskać dostęp do
przetworzonej chronionej informacji.

\textbf{Priorytet}: Wysoki

Niech w usłudze będą przechowywane pliki o znanych użytkownikowi
identyfikatorach.

Kiedy użytkownik zażąda pobrania agregatu plików o wybranych
identyfikatorach

Wtedy usługa dostarcza na lokalną maszynę użytkownika plik składający
się z części wspólnej linijek wybranych plików.


\subsection{Perspektywa administratora}

\subsubsection{Inicjalizacja danych w usłudze}
Jako administrator chcę pobrać zasoby z zewnętrznego źródła do plików
przechowywanych w usłudze aby umożliwić użytkownikom korzystanie z nich.

\textbf{Priorytet}: Wysoki


Niech pod pewnym adresem URL istnieje zasób dostępny przez HTTP lub
HTTPs.

Kiedy użytkownik wywoła z linii poleceń komendę klienta ,,\texttt{rsoctl
fetch <url>}'',

Wtedy klient zwraca identyfikator pliku, który użytkownik może użyć do
pobrania pliku z usługi.

\subsubsection{Dodanie węzła warstwy wewnętrznej}
Jako administrator chcę dodać nowy węzeł magazynowy aby zwiększyć
pojemność systemu.

\textbf{Priorytet}: Niski

Zakładając, że konfiguracja zawiera już jakąś listę węzłów warstwy
wewnętrznej,

Kiedy administrator doda do niej nową pozycję

Wtedy usługa pozwala nowemu węzłowi na dołączenie do sieci, zachowując
spójność przechowywanych danych.

\subsubsection{Dodanie węzła warstwy zewnętrznej}
Jako administrator chcę dodać nowy węzeł dostępowy aby zwiększyć przepustowość systemu.

\textbf{Priorytet}: Niski

Zakładając, że konfiguracja zawiera już jakąś listę węzłów warstwy zewnętrznej,

Kiedy administrator doda do niej nową pozycję

Wtedy nowy węzeł daje użytkownikom dostęp do usługi równoważny z dostępem przez pozostałe węzły.

\subsubsection{Uruchomienie węzłów usługi}
Jako administrator chcę dodać nowy węzeł dostępowy aby zwiększyć przepustowość systemu.

\textbf{Priorytet}: Niski

Zakładając, że konfiguracja zawiera już jakąś listę węzłów warstwy zewnętrznej,

Kiedy administrator doda do niej nową pozycję

Wtedy nowy węzeł daje użytkownikom dostęp do usługi równoważny z dostępem przez pozostałe węzły.

\subsubsection{Zatrzymanie węzłów usługi}
Jako administrator chcę móc zatrzymać usługę na wszystkich węzłach jedną komendą aby nie musieć zarządzać każdym węzłem indywidualnie.

\textbf{Priorytet}: Niski

Zakładając, że usługa jest uruchomiona

Oraz opisane w konfiguracji węzły są dostępne w sieci

Kiedy administrator wywoła na węźle komendę ,,\texttt{rsoctl stop}''

Wtedy usługa zostanie zatrzymana na wszystkich węzłach zdefiniowanych w konfiguracji

\subsubsection{Sprawdzanie stanu węzłów usługi}
Jako administrator chcę znać stan wszystkich węzłów usługi aby być w
stanie rozpoznać problemy i wiedzieć, czy konieczne jest podjęcie
działania w celu ich rozwiązania.

\textbf{Priorytet}: Wysoki

Zakładając, że usługa jest uruchomiona

Oraz opisane w konfiguracji węzły są dostępne w sieci

Kiedy administrator wywoła na węźle komendę ,,\texttt{rsoctl status}''

Wtedy zostaną zebrane i wyświetlone informacje o stanie wszystkich
węzłów usługi.

\section{Awarie}
W czasie działania usługi może wystąpić szereg awarii. Usługa powinna
pozostać dostępna, tzn. kontynuować poprawne świadczenie funkcjonalności
opisanych w poprzedniej sekcji w przypadku wystąpienia tych awarii oraz
ich kombinacji. Odporność na awarie może być testowana przez
wstrzykiwanie usterek w trakcie przebiegu podstawowych testów
funkcjonalnych.

Niech:

\[ n = \mbox{liczba węzłów magazynowych} \]
\[ i, j, k \leq \mbox{liczba replik podana w konfiguracji} - 1 < n \]

\subsection{Awaria węzłów}
Następuje nieplanowane wyłączenie $k$ z $n$ węzłów warstwy wewnętrznej.

\textbf{Priorytet}: Średni

\subsection{Awaria połączeń pomiędzy warstwą zewnętrzną i wewnętrzną}
Węzeł warstwy zewnętrznej nie może połączyć się lub traci nawiązane
połączenie z $i$ kolejnymi węzłami warstwy wewnętrznej.

\textbf{Priorytet}: Średni

\subsection{Awaria połączeń pomiędzy węzłami warstwy wewnętrznej}
Węzeł warstwy wewnętrznej nie może połączyć się lub traci nawiązane
połączenie z $j$ kolejnymi węzłami warstwy wewnętrznej.

\textbf{Priorytet}: Średni


\section{Działania niepożądane}
Wymagane jest, żeby usługa była niepodatna na następujące wektory ataku:

\subsection{Włączenie nieuprawnionego węzła}
Atakujący może próbować podstawić pod czasowo nieosiągalny (na przykład
w wyniku celowego przerwania łączności lub utraty dostępności w wyniku
ataku typu Denial of Service) węzeł własny, wrogi węzeł i w ten sposób
uzyskać kontrolę nad systemem lub uzyskać dostęp do chronionej
informacji. Nie rozważamy sytuacji, gdy administrator usługi
współpracuje (świadomie i dobrowolnie bądź nie) z atakującym.

\textbf{Priorytet}: Średni

\subsection{Dostęp przez nieautoryzowanego użytkownika}
Atakujący może próbować uzyskać dostęp do którego nie ma uprawnień z
poziomu zewnętrznego klienta. Może próbować to osiągnąć: polegając
bezpośrednio na niezabezpieczonym API usługi; monitorując ruch pomiędzy
uprawnionym użytkownikiem i usługą; uzyskując nieautoryzowany dostęp do
maszyny uprawnionego użytkownika. Nie rozważamy sytuacji, gdy uprawniony
użytkownik usługi współpracuje (świadomie i dobrowolnie bądź nie) z
atakującym.

\textbf{Priorytet}: Niski



\end{document}
