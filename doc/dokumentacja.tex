\documentclass[11pt,twocolumn]{article}
\usepackage[utf8]{inputenc}
\usepackage[polish]{babel}
\usepackage[QX]{fontenc}
\usepackage{lmodern}
\usepackage{textcomp}
\usepackage{graphicx}
\usepackage{tabu}
\usepackage{lipsum}

 

\title{}
\author{
  Daniel Danilewicz \\
  Maciej Kucharski \\
  Marek Dzienisiuk \\
  Paweł Stiasny (kierownik) \\
  Piotr Dziewicki \\
  Tomasz Pęksa \\
  Wiktor Garbacz \\
}


\begin{document}

\begin{titlepage}
  \centering
  \includegraphics[scale=0.3]{logopw.pdf} \par
  \vspace{0.5cm}

  {\Large Projekt RSO \par 2016L \par}
  \vspace{1cm}
  {\LARGE ,,Usługa bezpiecznej niezawodnej dystrybucji przetworzonej chronionej
          informacji'' \par}

  \vspace{2cm}

  Daniel Danilewicz \par
  Marek Dzienisiuk \par
  Piotr Dziewicki \par
  Wiktor Garbacz \par
  Maciej Kucharski \par
  Tomasz Pęksa \par
  Paweł Stiasny \par

  \vfill
\end{titlepage}


\begin{table*}[t]
  \begin{tabu}{ X[-1] X }
    \# & Usługa powinna: \\
    \hline
    RQ1 & Mieć dokładnie jeden, ten sam tekstowy plik konfiguracyjny dla części
      serwerowej i klientów usługi, \\ \hline
    RQ2 & Składać się po stronie serwerowej z dwóch warstw: (1)  wewnętrznej 
      wytwarzającej, przechowującej i przesyłającej przetworzoną wrażliwą
      informację, (2) warstwy zewnętrznej do udostępniania przetworzonej informacji
      oprogramowaniu klienckiemu usługi, \\ \hline
    RQ3 & Oprogramowanie realizujące część serwerową powinno być współbieżne, \\
      \hline
    RQ4 & Składać się po stronie klienckiej z prostych narzędzi opatulających
      zaproponowane API protokołu komunikacyjnego, \\ \hline
    RQ5 & Zapewniać odporność na uszkodzenia węzła, stąd każda z warstw powinna być
      uruchomiona na co najmniej dwóch węzłach, \\ \hline
    RQ6 & Zapewniać realizację usługi w trakcie awarii w czasie obsługi, \\ \hline
    RQ7 & Obsługiwać scenariusz próby ponownego wpięcia się przez węzeł dowolnej z
      warstw, z którym wcześniej utracono łączność, a zarazem być odporną na próbę
      zastąpienia nieosiągalnego sieciowo węzła (np. w wyniku ataku DoS, odmowa
      usługi) węzłem wrogim do tego nieuprawnionym, \\ \hline
    RQ8 & Na bieżąco i transparentnie dla oprogramowania klienckiego zarządzać
      dostępnymi zasobami lokalizacyjnymi, \\ \hline
    RQ9 & Zapewniać zadany (większy niż 1 ale dopuszczalny konfiguracją mniejszy niż
      liczba serwerów danych) poziom redundancji - z obsługą uzupełniania kopii na
      innych serwerach w przypadku awarii włącznie, \\ \hline
    RQ10 & Cała część sewerowa usługi powinna być startowana i zamykana jednokrotnym
      wywołaniem skyptu na jednym węźle serwera z jednym argumentem
      \textlangle{}start \textbar{} stop \textbar{} status\textrangle{}
      (wykorzystanie nieinteraktywnych metod automatycznego uwierzytelniania
      węzłów w SSH), \\ \hline
  \end{tabu}
  \caption{Lista wymagań}
  \label{table:lwym}
\end{table*}
 

\section{Wymagania}

W ramach projektu ma powstać dwuwarstwowa usługa.  Usługa ma spełniać
wymagania opisane w tabeli \ref{table:lwym}.

\lipsum

\end{document}
