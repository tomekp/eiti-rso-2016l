\documentclass[a4paper,11pt]{article}

\usepackage{polski}
\usepackage[utf8]{inputenc}
\usepackage{graphicx}
\usepackage{enumitem}

%opening
\author{
  Daniel Danilewicz \\
  Marek Dzienisiuk \\
  Piotr Dziewicki \\
  Wiktor Garbacz \\
  Maciej Kucharski \\
  Tomasz Pęksa \\
  Paweł Stiasny
}
\date{21 marca 2016}


\begin{document}

\makeatletter
\begin{titlepage}

\begin{center}
  \includegraphics[scale=0.1]{Logo_PW_black.png} \\
  \vspace{50pt}
  
  {\LARGE Usługa bezpiecznej, niezawodnej dystrybucji przetworzonej, chronionej informacji} \\
  \vspace{10pt}
  
  {\large Projekt realizowany w ramach przedmiotu RSO} \\
  \vspace{40pt}
  
  {\Huge Protokół ze spotkania \#1} \\
  \vspace{30pt}
  
  {\@author} \\
  \vspace{10pt}
  
  {\@date}
\end{center}

\end{titlepage}
\makeatother

\section{Wprowadzenie.}
  Pierwsze spotkanie projektowe odbyło się 21 marca 2016 r. na wydziale Elektroniki i Technik Informacyjnych Politechniki Warszawskiej.
  W spotkaniu uczestniczyli wszyscy członkowie zespołu.

\section{Agenda spotkania.}
  Spotkanie zostało podzielone na trzy sekcje tematyczne, tj.:
  
  \begin{enumerate}
    \item Wybór ról projektowych.
    \item Wstępna dyskusja nt. kształtu rozwiązania.
    \item Ustalenie trybu pracy.
  \end{enumerate}
  
  Każdy temat został oddzielnie poruszony i poddany dyskusji.
  W kolejnych podrozdziałach znajduje się szczegółowy opis wspólnych postanowień zespołu.

\subsection{Role projektowe.}
  Podczas spotkania udało się zidentyfikować, a następnie powierzyć role projektowe konkretnym członkom zespołu:
  
  \setlist[description]{leftmargin=\parindent,labelindent=\parindent}
  \begin{description}
    \item[Kierownik] - Paweł Stiasny
    \item[Architekt] - Tomasz Pęksa
    \item[Zarządca repozytorium] - Marek Dzienisiuk
    \item[Protokolant] - Piotr Dziewicki
    \item[Tester] - Maciej Kucharski
    \item[Handlowiec] - Maciej Kucharski
    \item[Specjalista ds. \textit{Docker}] - Tomasz Pęksa
    \item[Specjalista ds. bezpieczeństwa] - Wiktor Garbacz
  \end{description}
  
\subsection{Ogólny kształt rozwiązania.}
  W trakcie spotkania przeprowadzono burzę mózgów na temat wizji oraz sposobu rozwiązania postawionego problemu.
  Poniżej przedstawiono jedynie najważniejsze z ustaleń, natomiast komplet wymagań zostanie zebrany w~oddzielnym dokumencie:
  
  \begin{itemize}
    \item wykorzystanie języków programowania \textit{Java} i \textit{Python};
    \item uwierzytelnianie węzłów poprzez znane klucze publiczne;
    \item wykorzystanie biblioteki \textit{cURL};
    \item ilość kopii danych definiowalna w globalnym pliku konfiguracyjnym;
    \item dane udostępnione warstwie klienckiej w sposób przezroczysty;
    \item opracowanie metody wykrywania żywotności węzła;
    \item zabezpieczenie przed atakiem typu \textit{DoS} - metoda \textit{proof of work}.
  \end{itemize}
  
  Ustalono również, że ostateczna decyzja co do formy projektowanego rozwiązania należy do głównego \textbf{architekta}.

\subsection{Tryb pracy.}
  W kwestii trybu pracy poruszono dwa nadrzędne tematy: częstość spotkań projektowych oraz sposób wprowadzania zmian do repozytorium.
  
\subsubsection{Spotkania projektowe.}
  Zespół postanowił zrezygnować z regularnych spotkań projektowych.
  Komunikacja w czasie trwania projektu będzie odbywała się zdalnie.
  W przypadku ryzyka niedotrzymania terminu bądź wyraźnej prośby \textbf{kierownika}, \textbf{kierownik} zespołu ma prawo zorganizować dodatkowe spotkanie projektowe.

\subsubsection{Sposób commitowania.}
  Ustalono, że jedynie osoba na stanowisku \textbf{zarządcy} repozytorium ma prawo wykonania komendy \textit{push}.
  Pozostali członkowie zespołu podczas pracy zobowiązują się kierować \textit{pull requesty} do \textbf{zarządcy}.
  Po wstępnym \textit{review} dostarczonego modułu, jeśli ocena kodu jest pozytywna \textbf{zarządca} samodzielnie wykonuje komendę \textit{push} do głównego rezpozytorium.
  
\end{document}